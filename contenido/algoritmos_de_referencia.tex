\subsection{Topologías encontrados en los campos de aprendizaje de maquinas }
\subsubsection{Modelos utilizando algoritmos de referencia.}

A continuación, se ocupan distintas herramientas o modelos auxiliares para
evaluar el comportamiento de los modelos físicos mas complejos en secciones
posteriores.  

Cada modelo enlistado, en la parte inferior, se extrajeron de las librerías o
toolkits de python tales como
\href{https://scikit-learn.org/stable/}{\textcolor{cyan}{scikit-learn }}

\begin{itemize}
    \item{Regresión}

        Regresores GBRT

        \begin{align*}
            \hat{y_i} = F_M(x_i)= \sum_{m=1}^{M} h_m(x_i)
        \end{align*}
       Donde: 
        \begin{itemize}
            \item $x_i$ = entradas del modelo 
            \item $y_i$ = predicciones
            \item $h_m$ = En el contexto de las librerías y documentaciones de
                algunas funciones de código abierto  son Weak Learners,
                dependerán del tipo de regresor que estemos utilizado
        \end{itemize}
    \item{Random forest}
    \item{Logistic Regression}
    \item{Linear Discriminant Analysis}
    \item{K-Nearest Neighbors}
    \item{Classification and Regression Trees}
    \item{Naive Bayes}
    \item{Support Vector Machines}

\end{itemize}

