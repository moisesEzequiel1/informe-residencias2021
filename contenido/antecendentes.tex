
\textbf{Redes neuronales para modelar predicción de heladas}
\cite{ovando2005redes} 
\par En este trabajo se desarrollaron modelos basados en redes neuronales del
tipo "backpropagation", para predecir la ocurrencia de heladas, a partir de
datos meteorológicos de temperatura, humedad relativa, nubosidad, dirección y
velocidad del viento. El entrenamiento y la validación de las redes se
realizaron utilizando 24 años de datos meteorológicos correspondientes a la
estación de Río Cuarto, Córdoba, Argentina, separados en 10 años como conjunto
de datos de entrenamiento y 14 como conjunto de datos de validación. Se
construyeron diferentes modelos para evaluar el comportamiento de las redes
cuando se usan distintos números de variables de entrada y/o neuronas en la
capa oculta y las probabilidades de aciertos en los resultados de predicción
para los mismos, al considerar distintas variables de entrada. En los modelos
realizados, el porcentaje de días con error de pronóstico fue de 2\%,
aproximadamente, para 14 años de aplicación; cuando se consideran días de
heladas efectivas no pronosticadas los porcentajes oscilan entre un 10\% y un
23\%, para el mismo período. Los resultados de la simulación muestran el buen
desempeño y la pertinencia general de esta metodología en la estimación de
fenómenos de comportamiento no lineal como las heladas.

\par\textbf{Diseño de un control electrónico automático para la concentración
de dióxido de carbono en un microclima de jitomate fundamentado en un sistema
dinámico}
\cite{ponce2015diseno} 
\par En el presente trabajo de tesis se consideran los modelos dinámicos del
cultivo de jitomate y del microclima, con la finalidad de obtener una ley de
control óptima que permita conocer el comportamiento óptimo de todas las
variables involucradas en el sistema conjunto microclima-cultivo, y así,
diseñar el dispositivo electrónico que controlará la concentración de dióxido
de carbono en microclimas de jitomate.  \par El modelo dinámico conjunto
microclima-cultivo está formado por las variables de estado involucradas en
ambos sistemas, estas variables son: biomasa de frutos, biomasa de
hojas,consumo de nutrientes y concentración de dióxido de carbono. A partir de
la teoría de control óptimo y del sistema conjunto se selecciona una función de
costos, la cual tiene la finalidad de minimizar el gasto por consumo de energía
y maximizar la producción total de jitomate, después se obtiene un sistema de
variables adjuntas que permite evaluar al sistema en las condiciones finales
deseadas. Este procedimiento proporciona una ley de control que deberá depender
de una o más variables.  Después se resuelve el sistema de variables de estado
junto con el sistema de variables adjuntas y el resultado es el comportamiento
óptimo de todas las variables. Se hace el estudio de dicho comportamiento, en
especial, el de la variable de estado de concentración de CO2, ya que, este
comportamiento será la señal de referencia que deberá seguir el sistema de
control electrónico.

\par\textbf{Un procedimiento efectivo para descomponer y modelar series
temporales en agricultura} 
\cite{aragon2018procedimiento} 
\par En este trabajo proponemos una forma innovadora de abordar casos reales de
predicción de la producción de cultivos en una cooperativa. Nuestro enfoque
consiste en la descomposición de la serie temporal original de los cultivos en
sub-series temporales según una serie de factores, con el objetivo de generar
un modelo predictivo del cultivo a partir de los modelos predictivos parciales
de las sub-series. El ajuste de los modelos se realiza mediante un conjunto de
técnicas estadísticas y de Aprendizaje Automático. Esta metodología se ha
comparado con una metodología intuitiva que consiste en una predicción directa
de las series temporales. Los resultados muestran que nuestro enfoque logra un
mejor rendimiento de predicción que la manera directa, por lo que aplicar una
metodología de descomposición es mas adecuada para este problema que la no
descomposición.

\par\textbf{Aplicación de técnicas de aprendizaje automático para la
agricultura altoandina de Perú} 
\cite{caceresaplicacion}
\par En este trabajo se presentan  y describen las técnicas utilizadas en el
campo del aprendizaje automático. 
