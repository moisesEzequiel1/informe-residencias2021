La arquitectura usada para la telemetría se encuentra definida en la figura
\ref{fig:arquitectura_general_monitor}

\begin{figure}[H]
    \centering
    \includesvg[width=\linewidth,
    inkscapelatex=false]{images/arquitectura_telemetria.svg}
    \caption{Arquitectura general}
    \label{fig:arquitectura_general_monitor}
\end{figure}


Durante la configuración de la arquitectura se hace uso de herramientas tales
como docker, docker-compose en entornos linux, usando instancias en la nube. 

\newpage
\subsubsection{Configuración MQTT broker}

\begin{flushleft}
  \textbf{Configuración del servicio: MQTT broker.}
\end{flushleft}

\begin{minted}{yaml}
  mosquitto: 
    image: eclipse-mosquitto:2 
    ports:
      - 1883:1883 
    volumes:
      - ./mosquitto/mosquitto.conf:/mosquitto/config/mosquitto.conf 
      - mosquitto_data:/mosquitto/data 
      - mosquitto_log:/mosquitto/log 
\end{minted}

En esta sección se definen las siguientes propiedades. 
\begin{enumerate}
  \item Nombre del servicio dentro del archivo docker-compose.yml.
  \item Vamos a utilizar la versión 2 de la imagen oficial eclipse-mosquitto.
  \item El broker MQTT estará aceptando peticiones en el puerto 1883.
  \item Creamos un volumen de tipo bind mount para enlazar el archivo de
    configuración mosquitto.conf que tenemos en nuestro directorio local
    mosquitto con el directorio /mosquitto/config/mosquitto.conf del broker
    MQTT.
  \item Creamos un volumen para almacenar los mensajes que se reciben.
  \item Creamos un volumen para almacenar los mensajes de log.
\end{enumerate}

Es necesario realizar una configuración dentro del archivo
./mosquitto/mosquitto.conf para poder admitir conexiones al broker. 
\begin{minted}{yaml}
listener 1883
allow_anonymous true
\end{minted}
Donde se indica lo siguiente: 
\begin{enumerate}
  \item{listener 1883: es para habilitar la interfaz y el puerto en escucha del
    borker}
  \item{allow\_anonymous: deshabilita los protocolos de autenticación. siendo
    necesario indicarlo en algunas versiones del broker. }
\end{enumerate}

\begin{flushleft}
  \textbf{Configuración del servicio: Telegraf}
\end{flushleft}
\begin{minted}{yaml}
  telegraf: 
    image: telegraf:1.18 
      volumes:
        - ./telegraf/telegraf.conf:/etc/telegraf/telegraf.conf 
      depends_on: 
        - influxdb
\end{minted}
\begin{enumerate}
  \item Nombre del servicio dentro del archivo docker-compose.yml.
  \item Utilizamos la imagen Docker telegraf que tiene la etiqueta 1.18.
  \item Creamos un volumen de tipo bind mount para enlazar el archivo de
    configuración telegraf.conf que tenemos en nuestro directorio local
    telegraf con el archivo /etc/telegraf/telegraf.conf del contenedor.
  \item Indicamos que este servicio depende del servicio influxdb y que no
    podrá iniciarse hasta que el servicio de influxdb se haya iniciado.
\end{enumerate}

A partir de crear el servicio, es necesario genera un archivo de configuración, 
\begin{minted}{bash}
docker run --rm telegraf telegraf config > telegraf.conf
\end{minted}
Una vez creado el archivo es de configuración es necesario alojarlo en como se
muestra a continuación. (./telegraf/telegraf.conf)
\begin{minted}{bash}
.(directorio base)
--docker-compose.yml
----mosquitto
     --mosquitto.conf
----telegraf
     --telegraf.conf
\end{minted}
Dentro del mismo se encuentran bastantes configuraciones, disponibles para todos
los servicios y protocolos que soporta telegraf, sin embargo para el propósito
de la aplicación solo es necesario configurar las siguientes secciones.

\begin{enumerate}
  \item inputs.mqtt\_consumer: En este se encuentran definidas todas las
    propiedades y configuraciones requeridas para realizar la conexion hacia el
    broker MQTT en este caso mosquitto-eclipse. 
  \item output.influxdb: Aloja las configuración relacionadas a la salida que e
    este generara.
\end{enumerate}

Para el caso de inputs.mqtt\_consumer, hay distintos parametros de
configuración.
\begin{enumerate}
  \item servers:En este es necesario indicar la url donde esta alojado el
    broker, para este proyecto dado que se esta haciendo uso de docker solo es
    necesario poner el nombre del servicio. 
  \item topics: indica los tópicos a los que el cliente(telegraf) se
    subscribirá. 
  \item data\_format: indica el formato del tipo de dato que estaremos
    recibiendo a travez del mqtt broker. 
\end{enumerate}
\begin{minted}{bash}
 [[inputs.mqtt_consumer]]
   servers = ["tcp://mosquitto:1883"] 

   topics = [
     "esp23/#" //indica que nos queremos subscribir  a todos los topics a
     //partir de esp32
   ]

   data_format = "influx" 
\end{minted}

Por otro lado, para el caso de output.influxdb, los parametros son los
siguientes:
\begin{enumerate}
  \item Urls: indica el url de donde esta alojada el servicio de influxdb,
    (igual que el caso anterior el url corresponde con el nombre del servicio
    en docker)
  \item Database: indica el nombre de la base de datos. 
  \item Skip\_database\_creation: valor booleano que define si se crearan o no,
    las bases de datos, generalmente la idea es tener la tabla definida con
    anterioridad. 
  \item Username: corresponde a los nombres de usuario declarados en el
    servicio de docker-compose. 
  \item Password:corresponde a las contraseñas declaradas en el
    servicio dentro del archivo docker-compose.yaml.
\end{enumerate}



\begin{flushleft}
  \textbf{Configuración del servicio: InfluxDB}
\end{flushleft}

\begin{minted}{yaml}
influxdb: 
    image: influxdb:1.8 
    ports:
      - 8086:8086 
    volumes:
      - influxdb_data:/var/lib/influxdb 
    environment:
      - INFLUXDB_DB=esp32 
      - INFLUXDB_ADMIN_USER=root 
      - INFLUXDB_ADMIN_PASSWORD=root 
      - INFLUXDB_HTTP_AUTH_ENABLED=true 
\end{minted}

\begin{enumerate}
  \item Nombre del servicio dentro del archivo docker-compose.yml.
  \item Utilizamos la imagen Docker influxdb que tiene la etiqueta 1.8.
  \item Este servicio utilizará el puerto 8086 de nuestra máquina local para
    enlazarlo con el puerto 8086 el contenedor.
  \item Creamos un volumen con el nombre influxdb\_data que estará enlazado con
    el directorio /var/lib/influxdb del contenedor.
  \item Nombre de la base de datos.
  \item Usuario de la base de datos.
  \item Contraseña del usuario de la base de datos.
  \item Habilitamos la autenticación básica HTTP.
\end{enumerate}

\begin{flushleft}
  \textbf{Configuración del servicio: Grafana}
\end{flushleft}
La configuración de grafana esta conpuesta con las siguientes bases.

\begin{minted}{yaml}
grafana: 
    image: grafana/grafana:7.4.0 
    ports:
      - 3000:3000 
    volumes:
      - grafana_data:/var/lib/grafana 
      - ./grafana-provisioning/:/etc/grafana/provisioning 
    depends_on: 
      - influxdb
\end{minted}

\begin{enumerate}
\item Nombre del servicio dentro del archivo docker-compose.yml.
\item Utilizamos la imagen Docker grafana.
\item Este servicio utilizará el puerto 3000 de nuestra máquina local para
  enlazarlo con el puerto 3000 el contenedor.
\item Creamos un volumen con el nombre grafana\_data que estará enlazado con el
  directorio /var/lib/grafana del contenedor.
\item Indicamos que este servicio depende del servicio influxdb y que no podrá
  iniciarse hasta que el servicio de influxdb se haya iniciado.
\end{enumerate}


Como parte importante es necesario configurar las fuentes de datos,
configuraciones generales, en este caso, se declaran a partir de archivos yml
en diferentes directorios dentro de el espacio de trabajo de grafana. 

Archivos de configuración son los siguientes:

El archivo inferior, corresponde al aprovisionamiento automatico de las fuentes
de datos, en el caso siguiente grafana cuenta con funciones directamente para
conectase a influxDB.
\begin{minted}{yaml}
apiVersion: 1
datasources:
  - name: InfluxDB
    type: influxdb
    access: proxy
    database: esp32 
    user: root 
    password: root 
    url: http://influxdb:8086 
    isDefault: true
    editable: true
\end{minted}
Parámetros: 
\begin{enumerate}
  \item Nombre de la base de datos InfluxDB
  \item Nombre de usuario para acceder a la base de datos
  \item Contraseña del usuario para acceder a la base de datos
  \item URL del servicio InfluxDB
\end{enumerate}

El archivo de configuración inferior, crea un aprovisionamiento automático de
dashboards, donde los mismos podrían ser definidos, base a la documentación de
grafana, revisando cada uno de los parámetros que se pueden agregar al
archivo.json del dashboard a crear.
    


\begin{minted}{yaml}
apiVersion: 1
providers:
- name: InfluxDB
  folder: ''
  type: file
  disableDeletion: false
  editable: true
  options:
    path: /etc/grafana/provisioning/dashboards
\end{minted}

Parámetros: 
\begin{enumerate}
  \item Nombre del servicio dentro del archivo docker-compose.yml.
  \item Utilizamos la imagen Docker grafana.
  \item Este servicio utilizará el puerto 3000 de nuestra máquina local para
    enlazarlo con el puerto 3000 el contenedor.
  \item Creamos un volumen con el nombre grafana\_data que estará enlazado con
    el directorio /var/lib/grafana del contenedor.
  \item Creamos un volumen de tipo bind mount entre el directorio local de
    nuestra máquina ./grafana-provisioning/ y el directorio
    /etc/grafana/provisioning del contenedor.
  \item Indicamos que este servicio depende del servicio influxdb y que no
    podrá iniciarse hasta que el servicio de influxdb se haya iniciado.
\end{enumerate}


A cada archivo le corresponde un sitio en el espacio de trabajo(directorio) 

\begin{minted}{yaml}
.(directorio de trabajo)
grafana-provisioning
---dashboards
  ----CO2Dashboard.json
  ----dashboard.yml
---datasources
  ----datasource.yml
\end{minted}

En la documentación ofical se pueden obtener datos o propiedades de una
dashboard por ejemplo.
\href{https://grafana.com/docs/grafana/latest/dashboards/}{grafana
docs}.



\subsubsection{Configuración general}
Durante la configuración de los servicios y instacias requeridas se parte de
los siguietes requerimientos. 
\begin{enumerate}
  \item Docker, docker-compose. \href{https://docs.docker.com/compose/}{guia}.
  \item alguna maquina, o instancia en la nube disponible para hostear los
    servicios.
\end{enumerate}

Partiendo de ello, se llevan es necesario, seguir todas las instrucciones
anteriores para poder implementar dicho monitor. 

En la parte superior se encuentra disponible el archivo docker-compose.yml.
\begin{minted}{yaml}
version: '3'

services:
  mosquitto:
    image: eclipse-mosquitto:2
    ports:
      - 1883:1883
    volumes:
      - ./mosquitto/mosquitto.conf:/mosquitto/config/mosquitto.conf
      - mosquitto_data:/mosquitto/data
      - mosquitto_log:/mosquitto/log

  telegraf:
    image: telegraf:1.18
    volumes:
      - ./telegraf/telegraf.conf:/etc/telegraf/telegraf.conf
    depends_on:
      - influxdb

  influxdb:
    image: influxdb:1.8
    ports:
      - 8086:8086
    volumes:
      - influxdb_data:/var/lib/influxdb
    environment:
      - INFLUXDB_DB=${INFLUXDB_DB}
      - INFLUXDB_ADMIN_USER=${INFLUXDB_USERNAME}
      - INFLUXDB_ADMIN_PASSWORD=${INFLUXDB_PASSWORD}
      - INFLUXDB_HTTP_AUTH_ENABLED=true

  grafana:
    image: grafana/grafana:7.4.0
    ports:
      - 3000:3000
    volumes:
      - grafana_data:/var/lib/grafana
      - ./grafana-provisioning/:/etc/grafana/provisioning
    depends_on:
      - influxdb

  chronograf:
    image: chronograf:1.8
    ports:
      - 8888:8888
    volumes:
      - chronograf_data:/var/lib/chronograf
    depends_on:
      - influxdb
    environment:
      - INFLUXDB_URL=http://influxdb:8086
      - INFLUXDB_USERNAME=${INFLUXDB_USERNAME}
      - INFLUXDB_PASSWORD=${INFLUXDB_PASSWORD}

volumes:
  mosquitto_data:
  mosquitto_log:
  node_red_user_data:
  influxdb_data:
  grafana_data:
  chronograf_data:
\end{minted}

Finalmente seria tan simple como usar el comando docker-compose up -d para
correr todos los servicios que involucran a la solucion, y declarar un par
variables de entorno (.env), serian las contraseñasde influxdb, usuario, nombre
de la base de datos, y las credenciales para grafana. todas estas se ecunetran
entre llave, con un signo de dolar al inicio dentro del archivo de
configuración.

\textbf{**Nota:} En el caso del proyecto en mencion se esta haciendo uso del
servicio ec2 de Amazon Web Services, para alojar la maquina virtual que soporta
los servicios. Siendo posible instanciar dichos servicios en un entorno local
con ciertas características**.

{\tiny ** una ip publica, en caso de no estar alojada esta en la misa seccion
de red.}



