
\begin{table}[H]
  \centering
  \resizebox{\textwidth}{!}{%
    \begin{tabular}{{p{10cm}}*{16}{m{0.2cm}}}\hline
      &\multicolumn{16}{c}{Semanas}\\\cmidrule{2-17}
      Actividad & 1& 2& 3& 4& 5& 6& 7& 8& 9& 10& 11& 12& 13& 14& 15& 16\\\hline
      Análisis de los requerimientos del proyecto y estudio de los recursos
      que se tiene para su desarrollo.
      &  x&  x&  x& & & & & & & & & & & & & \\\midrule
      Realizar el acondicionamiento del sistema de humedad de suelo en el
      micro-invernadero, con el fin de tener las condiciones adecuadas para la
      captura de datos reales del sistema para el diseño del modelo por redes
      neuronales.\vspace{2mm}  Desarrollar los temas relacionados al modelado en sistemas
      de tipo invernadero. A su vez se comenzará a evaluar la integridad del
      cronograma propuesto, y estructuración del informe final, buscando
      realizar las primeras revisiones de estos (informe y cronograma).
      & & & & x& x& & & & & & & & & & & \\\midrule
      Durante estas semanas comenzaremos a documentar características que
      definen a los sistemas de redes neuronales el resultado esperado de las
      primeras semanas (4,5,6,7,8), corresponde a localizar características,
      ventajas, conceptos prácticos aplicables, detección de entradas y
      salidas, a controlar dentro del sistema invernadero.  Partiendo de la
      documentación revisada, procedo a diseñar un modelo de redes neuronales
      utilizando estructuras conocidas (MLP modelo de perceptrón multicapas o
      modelos de capas densas) que representen al sistema físico, empleando las
      tecnologías que se localicen durante la investigación relacionadas al
      campo de inteligencia artificial, con anterioridad se han estado
      revisando distintas herramientas tales como Simulink (empleando toolkits
      para modelar rápidamente), herramientas enfocadas ( keras, tensorflow,
      pytorch) muchas de estas empleadas en los campos de IA con un gran
      impacto en la actualidad. \vspace{2mm}

      A partir del diseño a presentar. Lo siguiente
      seria simular los modelos de redes neuronales, validando así los mimos.
      Con el fin de estudiar y verificar toda la información obtenida, durante
      el proceso anterior.
      & & & & x& x& x& x& x& x& x& x& x& & & &\\\midrule
      Se finalizará el proceso de escritura del informe final para su entrega.
      & & & & & & & & & & & & & x& x& x& x\\\hline
  \end{tabular}}
  \caption{Cronograma de actividades esperado}
  \label{t:cronograma}
\end{table}

\textbf{Nota: }Durante el periodo de residencias ago.-dic. 2021, se busca estar revisando de
manera periódica (cada mes), el informe final, mismo que tiene como finalidad
ser la introducción a mi proyecto de tesis.
